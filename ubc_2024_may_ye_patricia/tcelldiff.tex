

Within the subgroup of T lymphocytes are CD8+ T cells and helper and regulatory CD4+ T cells. Each of these subsets encapsulate a number of additional subsets. CD8+ T cells destroy virus-infected cells and tumour cells and produce anti-viral cytokines such as interferon gamma \cite{Sigal2016} while the primary function of helper CD4+ T cells is to assist other immune cells in immune responses and regulatory CD4+ T cells maintain immune homeostasis \cite{Xia2012}. 
The differentiation of T cells begins as common lymphoid progenitors (CLPs) migrate from bone marrow to the thymus. In the thymus, the CLPs become double-negative (DN) thymocytes, they do not express the T-cell co-receptors CD4 and CD8, nor T cell receptor (TCR). Following the expression of pre-TCR, the DN T cells transition to double-positive (DP) where both CD4 and CD8 are expressed and pre-TCR is replaced with TCR \cite{Germain2002}.

Signaling mediated by TCR interaction with self-peptide presented by major histocompatibility complex (MHC) classes play a large role in determining the fate of the DP T cells. Those that bind to self-peptide-MHCI complexes become CD8+ T cells, losing CD4 expression, while those that bind to self-peptide-MHCII ligands become naïve CD4+ T cells, losing CD8 expression. These single-positive T cells are then able to migrate to peripheral lymphoid sites from the thymus. 