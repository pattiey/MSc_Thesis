\section{Summary of Findings}
In this thesis, we present LaRCH, a neural topic model with a built-in node tree structure that enforces hierarchical relationships between latent topics through shared additive embedding features. A spike-and-slab prior informed gene embedding model parameters and provides a natural method of marker gene selection and the ability to perform interpretable latent feature detection. From multiple tests on realistic simulated scRNA-seq datasets, we show that LaRCH is effective at cell type discrimination in heterogeneous samples of immune cell gene expression profiles. By enforcing an underlying topic structure, we allow for an intuitive representation of nested relationships between cell types with shared lineage paths. 

Application of LaRCH on a single-cell gene expression dataset across multiple disease phenotypes of systemic lupus erythematosus shows disease dependent hierarchical genomic features. Our findings show cell type specific differences in gene expression in individuals with SLE in addition to overarching cellular features present across all cell types, suggesting disease mediated altered immune function. Investigation into specific node-level marker genes shows the detection of canonical immune subset gene markers and genes with significant roles in immune function. Specifically, we detect differences in the expression of APOBEC3A in monocyte cell populations and PRKQ-AS1 in CD4+ T lymphocyte populations. APOBEC3A is an enzyme encoding gene that has been shown to be upregulated in patients with SLE, while PRKQ-AS1 is a lncRNA that has recently been associated with aberrant immune function and may play a key role in immunological disorders. Both genes show potential impact on the disease mechanisms of SLE and other autoimmunological disorders and are promising avenues for further investigation.

\section{Limitations}

The use cases of the LaRCH model are limited to samples of heterogeneous cell populations with highly variable gene features. Granularity of cell type detection proves to be a challenge and cell subsets with large overlapping gene expression profiles are difficult to discern from one another. Training a model with a deeper latent tree did not yield better results in this aspect. This means that the application of this model would not be suitable for several uses cases such as constructing cancer phylogenies or the detection of rare cell subsets, where detection of minute differences is necessary.
 
Using a fixed perfect binary tree structure may lead to over parameterization of the model where many components of the tree are unused, resulting in an ill-fitting tree structure to represent the data. Unfortunately, including a tree-fitting component to the model would greatly increase computational burden as the search space would be super-exponential. \textit{Post-hoc} tree pruning could help solve this problem, but would result in the loss of information in the model. In LaRCH, we combat over-fitting at the embedding parameter level by implementing Bayesian priors that encourage model sparsity.

Training of the LaRCH model requires intensive computational resources. LaRCH is implemented in Python using a machine-learning library (\texttt{PyTorch} \cite{pytorch}), which often has specialized requirements for hardware, such as graphics processing units (GPUs), with sufficient processing capabilities and memory capacity. All model training experiments presented in this thesis were executed on a Nvidia Linux X64 Display Driver GPU. Smaller scRNA-seq datasets of 2,000 cells across ~50,000 genes requires on average 12 minutes to complete model training with a tree depth of 5 in 1,000 epochs. However, on a large gene expression dataset of 1.2 million cells across ~30,000 genes, training required over 60 hours to complete.

\section{Future Directions}
The LaRCH model may be readily expanded to include other HTS data types. Integration of ChIP-seq data of transcription factor binding affinity or ATAC-seq data of DNA accessibility profiles would be natural additions to this model. Multi-omic integration could provide valuable insights into the gene mechanistic drivers of cell type differentiation. 

To understand overall trends in up and down-regulated gene sets, one could relax element-wise sparsity in facour of set-wise sparsity and fine-mapping of genes, building upon the Sum of Single Effects (SuSiE) method of variable selection outlined in Wang \textit{et al.} \cite{susie}. Such a formulation would likely result in a more scalable iterative coordinate-wise algorithm, while assigning high probability mass on causal/anchor features for each independent effect. In fact, implementing a tree-structured latent feature space is also a promising extension to less computationally intensive factorization-based learning algorithms such as NMF. Such algorithms are able to yield low-dimensional embeddings of genomic data, maintaining many use cases of a neural topic model, in an more computationally efficient manner. 

Finally, using the results of the LaRCH model, a number of validation experiments can be performed. From the genes described in Chapter \ref{cha:genes}, genetic perturbation experiments on healthy and disease models to uncover the downstream effects of differential expression of these genes. From these experiments, there is a potential for bettering our understanding of mechanisms of immune function involved in driving the pathogenesis of autoimmune disorders including SLE.