\section{CD4+ T Cell isolation}
\label{cha:cd4diff}
For \textit{in vitro} cell culture, male wild-type BL6 mice were bred and kept in specific pathogen-free (SPF) conditions at the Centre for Molecular Medicine and Therapeutics (CMMT) mouse facility at BC Children's Hospital Research Institute. All animals were cared for in compliance with the Canadian Council on Animal care and the University of British Columbia Animal Care Committee (Protocol numbers A19-024, A19-0273 and A21-0266). 

Spleen and lymph nodes were harvested from mice at 8-14 weeks of age. Following organ harvest, naïve CD4+ T cells were isolated by negative selection using a StemCell Mouse CD4+ T Cell Isolation Kit according to the manufacturer's protocol (StemCell Technologies, Cat: 19852). 

\section{CD4+ T Cell Subtype Polarization}
All isolated CD4+ T cells were cultured in T cell media (TCM) containing RPMI 1640 media (Corning) supplemented with 10\% fetal bovine serum (FBS) (Gibco), 2mM L-Glutamine, 100 U/mL penicillin/streptomycin, 55 $\mu$M $\beta$-Mercaptoethanol (Life Technologies). Cells were activated using 5$\mu$g/mL plate-bound anti-CD3 (BioXCell), 2 $\mu$g/mL soluble anti-CD8 (BioXCell) and 10 ng/mL recombinant human (rh) IL-2 (Peprotech) at a concentration of $1.5 \times 10^6$ cells/mL. 

In addition to the above media, Helper T cell subtypes were polarized as follows: 
\begin{itemize}
    \item T$_H$0: No additional cytokines or antibodies
    \item T$_H$1: 4 $\mu$g/mL anti-mouse IL-4 (BioXCell), 10 ng/mL recombinant murine (rm) IL-12
    \item Regulatory T cell (T$_\text{reg}$): 10 ng/mL rh TGF-$\beta$1 (Peprotech)
    \item Non-pathogenic T$_H$17: 10 ng/mL rmIL-6 (Peprotech), 5 ng/mL TGF-$\beta$1
    \item Pathogenic T$_H$17 (T$_H$17p): 10 ng/mL rmIL-6, 5 ng/mL TGF-$\beta$1, 10 ng/mL rmIL-1$\beta$ (Peprotech), 10 ng/mL rmIL-23 (BioLegend)
\end{itemize}

72-hours after initial activation, cells were split with fresh media containing the same polarization conditions with the exception anti-CD3 and anti-CD28 which were omitted. On day 4, cells were collected for downstream flow cytometry analysis. 

\section{Flow Cytometry Analysis}
Cells collected for flow cytometry analysis are spun, supernatant discarded, and resuspended in TCM supplemented with 10 ng/mL rhIL-2 and 1.5 $\mu$g/mL Brefeldin A (eBioscience). A subsample of cells were further restimulated with phorbol 12-myristate (PMA) (25 ng/mL; Sigma) and ionomycin (250 ng/mL; Sigma). Following resuspension, all samples are incubated for 4 hours before staining. 

Surface proteins are stained by resuspending cell samples in FACS buffer comprised of Phosphate Buffered Saline (PBS) (Corning), 2\% FBS, and surface marker antibody cocktail containing PerCP-Cy5.5 anti-mouse CD69 (1:200, clone: H1.2F3 BioLegend), FITC anti-mouse CD25 (1:200, clone: PC61 BioLegend), eFluor 780 Fixable Viability Dye (1:10000, eBioscience), BV785 anti-mouse CD8$\alpha$ (1:200, clone: 53-6.7, BioLegend), and BV605 anti-mouse CD4 (1:200, clone: RM4-4, BioLegend). Cells are incubated in antibody cocktail for 30 minutes at room temperature in the dark. 

To detect intracellular proteins, all samples were fixed and permeabilized using the Foxp3/Transcription Factor Staining Buffer Set (eBiosciences), followed by intracellular marker antibody cocktail containing Alexa Fluor 700 monoclonal antibody FOXP3 (1:200, clone: FJK-16s, eBiosciences), BV510 anti-mouse ROR$\gamma$t (1:200, clone: Q31-378), BV421 anti-T-bet (1:200, clone: 4B10, BioLegend), APC anti-mouse IL-17A (1:200, clone: TC11-18H10.1, BioLegend), and PE IFN-$\gamma$ (1:200, clone: XMG1.2, BioLegend). Cells are incubated for 60 minutes at room temperature in the dark. Sample were resuspended in FACS buffer and FACS data was acquired with a BD LSRFortessa X-20 flow cytometer. FlowJo software was used for analysis. 
